
\documentclass[letterpaper,oneside]{article}
\usepackage[margin=1in]{geometry}
\usepackage{tabls}
\usepackage{booktabs}
\usepackage{amsmath}
\usepackage{amssymb}
\usepackage{amsthm}
\usepackage{amsfonts}
\usepackage{enumitem}
\usepackage{microtype}
% Abbreviations

\newcommand{\ii}{\ensuremath{\mathbf{\hat \i}}}
\newcommand{\jj}{\ensuremath{\mathbf{\hat \j}}}
\newcommand{\kk}{\ensuremath{\mathbf{\hat k}}}
\newcommand{\vv}{\ensuremath{\mathbf{v}}}
\newcommand{\colvec}[1]{\ensuremath{\begin{bmatrix}#1\end{bmatrix}}}
\DeclareMathOperator{\rank}{rank}
\DeclareMathOperator{\rref}{rref}
\DeclareMathOperator{\vspan}{span}
\DeclareMathOperator{\trace}{tr}
\DeclareMathOperator{\col}{col}
\DeclareMathOperator{\row}{row}
\DeclareMathOperator{\nullspace}{null}
\DeclareMathOperator{\adjoint}{adj} % adjoint
\DeclareMathOperator{\diag}{diag}
\newcommand{\RR}{\ensuremath{\mathbb{R}}}

\renewcommand{\thefootnote}{\roman{footnote}}
\newcommand{\note}[1]{\footnote{#1}\marginpar{\fbox{\textbf{\thefootnote}}}}
%%%%%%%To make all notes ignored, uncomment the following command.
\renewcommand{\note}[1]{}



\newcommand{\cl}[1]{  \begin{matrix}  #1  \end{matrix}  }
\newcommand{\bm}[1]{  \begin{bmatrix}  #1  \end{bmatrix}  }
\newcommand{\inv}{^{-1}}
\newcommand{\im}{\ensuremath{\text{im }}}
\newcommand{\R}{\mathbb{R}}
\providecommand{\norm}[1]{\lVert#1\rVert} 
\newcommand{\st}{\ensuremath{\mid}}


%------------------------------------------------------------------------------------------------------------

\usepackage[pdfpagelabels]{hyperref}
\makeindex

\begin{document}

%\frontmatter
\title{Linear Algebra Objectives}
\maketitle

Here is a list of objectives for the book.  Notice that many
objectives are simple extensions of other objectives and involve the
same main ideas.  

\begin{itemize}
\item Chapter 1
\input{01-Arithmetic/Arithmetic-Objectives}
\item Chapter 2
\input{02-Applications/Linear-Algebra-Applications-Objectives}
\item Chapter 3
\input{03-Patterns/Patterns-Objectives}
\item Chapter 4
\input{04-Inner-Products/Inner-Products-Objectives}
\item Chapter 5
\input{05-Linear-Transformations/Linear-Transformations-Objectives}
\item Chapter 6
\input{06-Changing-Bases/Changing-Bases-Objectives}
%\item Chapter 7
%\input{07-Jordan-Form/Jordan-Form-Objectives}
\end{itemize}


\end{document}


%%% Local Variables: 
%%% mode: latex
%%% TeX-master: t
%%% End: 

