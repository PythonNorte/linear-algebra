
\begin{enumerate}

\item Construct graphical displays of linear transformations.  Graphically explain how to see eigenvalues, eigenvectors, and determinants from these visualizations, as well as when an inverse transformation exists.
\item Know the vocabulary of functions: domain, range, image, inverse image, kernel, injective (one-to-one), surjective (onto), and bijective. 
\item Define linear transformation, and give examples involving both finite and infinite dimensional vector spaces. 
Show that every linear transformation between finite dimensional vector spaces can be represented by a matrix in relation to the standard basis vectors.
\item Describe the column space, null space, and eigenspaces of a matrix. Show that these spaces are vector spaces. 
\item Show that every solution to the linear equation $T(\vec x)=\vec b$ can be written in the form $\vec x = \vec x_p+\vec x_h$, where $\vec x_p$ is a particular solution and $\vec x_h$ is a solution to the homogeneous equation $T(\vec x)=\vec 0$.
\item Explain what it means to compose linear transformations (both algebraically and graphically). Then show how to decompose a matrix as the product of elementary matrices, and use this decomposition (through row reduction) to compute the determinants of large matrices. 


\end{enumerate}
