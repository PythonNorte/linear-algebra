\section{Preparation}

\noindent
This chapter covers the following ideas. When you create your lesson plan, it should contain examples which illustrate these key ideas. Before you take the quiz on this unit, meet with another student out of class and teach each other from the examples on your lesson plan. 


\begin{enumerate}

\item Construct graphical displays of linear transformations.  Graphically explain how to see eigenvalues, eigenvectors, and determinants from these visualizations, as well as when an inverse transformation exists.
\item Know the vocabulary of functions: domain, range, image, inverse image, kernel, injective (one-to-one), surjective (onto), and bijective. 
\item Define linear transformation, and give examples involving both finite and infinite dimensional vector spaces. 
Show that every linear transformation between finite dimensional vector spaces can be represented by a matrix in relation to the standard basis vectors.
\item Describe the column space, null space, and eigenspaces of a matrix. Show that these spaces are vector spaces. 
\item Show that every solution to the linear equation $T(\vec x)=\vec b$ can be written in the form $\vec x = \vec x_p+\vec x_h$, where $\vec x_p$ is a particular solution and $\vec x_h$ is a solution to the homogeneous equation $T(\vec x)=\vec 0$.
\item Explain what it means to compose linear transformations (both algebraically and graphically). Then show how to decompose a matrix as the product of elementary matrices, and use this decomposition (through row reduction) to compute the determinants of large matrices. 


\end{enumerate}


Here are the preparation problems for this unit.  Problems that come from Schaum's Outlines ``Beginning Linear Algebra'' are preceded by a chapter number. From here on out, many of the problems we will be working on come from Schaum's Outlines.  Realize that sometimes the method of solving the problem in Schaum's Outlines will differ from how we solve the problem in class. The difference is that in Schaum's Outlines they almost always place vectors in rows prior to row reduction, whereas we will be placing vectors in columns. There are pros and cons to both methods. Using columns helps us focus on understanding coordinates of vectors relative to bases, whereas using rows helps us determine when two vector spaces are the same.


















\begin{center}
\begin{tabular}{ll|l}
\multicolumn{2}{c}{Preparation Problems (\href{http://ilearn.byui.edu/bbcswebdav/institution/Physical\_Sci\_Eng/Mathematics/Personal\%20Folders/WoodruffB/341/4-Linear-Transformations-Preparation-Solutions.pdf}{click for handwritten solutions})}
%&
%Webcasts 
%(
%\href{http://ilearn.byui.edu/bbcswebdav/institution/Physical\_Sci\_Eng/Mathematics/Personal\%20Folders/WoodruffB/341/4-Linear-Transformations-videos.pdf}{pdf copy}
%)
\\
\hline\hline
Day 1& 
1a, 
2a, 
Schaum's 8.4, 
Schaum's 8.14
%&
%\href{http://ilearn.byui.edu/bbcswebdav/institution/Physical\_Sci\_Eng/Mathematics/Personal\%20Folders/WoodruffB/341/4-Linear-Transformations-video-01.wmv}{1},
%\href{http://ilearn.byui.edu/bbcswebdav/institution/Physical\_Sci\_Eng/Mathematics/Personal\%20Folders/WoodruffB/341/4-Linear-Transformations-video-02.wmv}{2},
%\href{http://ilearn.byui.edu/bbcswebdav/institution/Physical\_Sci\_Eng/Mathematics/Personal\%20Folders/WoodruffB/341/4-Linear-Transformations-video-03.wmv}{3}
\\ \hline
Day 2& 
Schaum's 8.17, 
Schaum's 9.7, 
Schaum's 8.21, 
Schaum's 8.29
%&
%\href{http://ilearn.byui.edu/bbcswebdav/institution/Physical\_Sci\_Eng/Mathematics/Personal\%20Folders/WoodruffB/341/4-Linear-Transformations-video-04.wmv}{4},
%\href{http://ilearn.byui.edu/bbcswebdav/institution/Physical\_Sci\_Eng/Mathematics/Personal\%20Folders/WoodruffB/341/4-Linear-Transformations-video-05.wmv}{5},
%\href{http://ilearn.byui.edu/bbcswebdav/institution/Physical\_Sci\_Eng/Mathematics/Personal\%20Folders/WoodruffB/341/4-Linear-Transformations-video-06.wmv}{6}
\\ \hline
Day 3& 
Schaum's 9.4, 
Schaum's 8.38, 
Schaum's 5.24,
Schaum's 3.15, 
%&
%\href{http://ilearn.byui.edu/bbcswebdav/institution/Physical\_Sci\_Eng/Mathematics/Personal\%20Folders/WoodruffB/341/4-Linear-Transformations-video-07.wmv}{7},
%\href{http://ilearn.byui.edu/bbcswebdav/institution/Physical\_Sci\_Eng/Mathematics/Personal\%20Folders/WoodruffB/341/4-Linear-Transformations-video-08.wmv}{8},
%\href{http://ilearn.byui.edu/bbcswebdav/institution/Physical\_Sci\_Eng/Mathematics/Personal\%20Folders/WoodruffB/341/4-Linear-Transformations-video-09.wmv}{9}
\\ \hline
Day 4&
Everyone do 4
&
\\ \hline
Day 5&
Lesson Plan,
Quiz, Start Project 
&
\\ \hline
\end{tabular}
\end{center}


\begin{center}
\begin{tabular}{|l|l|l|l|l|}
\hline
Concept&Where&Suggestions&Relevant Problems\\ \hline
Visualizations &Here&1ac,2ab,3ab&All\\ \hline
Mappings&Schaum's Ch 8&1,4,7,9,53&1-10,12,51-55\\ \hline
Linear Transformations&Schaum's Ch 8&14,15,17&13-20, 57-66\\ \hline
Standard Matrix Representation&Schaum's Ch 9&4,7,10&1a,4,7,10,27a,29,30,34,42,\\ \hline
Important subspaces&Schaum's Ch 8&21,23,25&21-26, 70-78\\ \hline
Types of Transformations&Schaum's Ch 8&29,31,35,38&29-32,34-39,79-80,85,89-91\\ \hline
Homogeneous Equations&Schaum's Ch 5&24,26&24-26,62-65\\ \hline
Elementary Matrices&Schaum's Ch 3&15,82&12-17,55-56,81-83\\ \hline
Determinants&Here&4&4 (multiple times)\\ \hline
\end{tabular}
\end{center}
