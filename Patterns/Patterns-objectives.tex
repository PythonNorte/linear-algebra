
\begin{enumerate}

\item Create examples to illustrate mathematical definitions. Know the definitions in this chapter.
\item Create examples to illustrate theorems. You do not have to memorize all the theorems in this chapter, but when one is given, make sure you can construct examples of it. 
\item When a matrix has an inverse, this is equivalent to many other facts about the matrix and corresponding systems of equations.  Illustrate this equivalence with examples, and be able to explain why the the ideas are equivalent.
\item Be able to explain why something is or is not a vector space. Give examples of vector spaces and subspaces that you have encountered previously in this class and other classes.
\item Explain how to find a basis for a vector space and the dimension of a vector space. Give the coordinates of vectors relative to a chosen basis.

\end{enumerate}
\note{Give an explicit list of definitions to know}
\note{Explicitly know: inverse matrix theorem, axioms of vector
  addition/scalar multiplication, subspace test, det(AB)=det(A)det(B)}
%%% Local Variables: 
%%% mode: latex
%%% TeX-master: "../linear-algebra"
%%% End: 
