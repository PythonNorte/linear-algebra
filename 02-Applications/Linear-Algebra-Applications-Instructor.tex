\section{Notes for teaching}

\subsection{Ben Woodruff}

Don't let yourself get bogged down with all the details in chapter 2. The students can quickly see why many of the ideas work, and you don't have to go through all the details. 

\subsubsection{Interpolating Polynomials and regression}

Try to do both interpolating polynomials and linear regression with the same data set, one right after the other.  Then all you have to do is say, ``well, we have to erase column 3, but now we have an over determined system.'' I suggest, ``if we could somehow reduce this to a system with only two equations, we'd be able to solve it.  If we multiply on the left on both sides by a 2 by 3 (or $n$) matrix, then we'd have only two rows.  Are there any 2 by $n$ matrices in the problems already? No, unless we transpose the coefficient matrix.  Let's try it and see what happens.''  Then after they try it, you can talk more about why it works.

\subsubsection{Derivatives}

You will find that the derivative section can bog your class down quickly if lots of students haven't had multiple calculus classes. Keep the examples simple, show them with pictures what's happening, and tell them they will learn more about it in other classes. 
