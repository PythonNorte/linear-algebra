\section{Preparation}

\noindent
This chapter covers the following ideas. When you create your lesson plan, it should contain examples which illustrate these key ideas. Before you take the quiz on this unit, meet with another student out of class and teach each other from the examples on your lesson plan. 


\begin{enumerate}

\item Explain the dot product and inner product. Use them to define length and angles of vectors. Normalize vectors, obtain vectors of a given length in a given direction, and explain how to tell if two vectors are orthogonal. 

\item Obtain the orthogonal complement of a set of vectors (relate it to the null space of the transpose). Use the orthogonal complement to project vectors onto vector spaces. 

\item By considering the projection of a vector onto a vector subspace, explain why solving $A^TA \vec x = A^T\vec b$ gives the solution to the least squares regression problem.

\item Use the Gram-Schmidt orthogonalization process to obtain an orthonormal basis of vectors. Show how to compute components of vectors relative to an orthogonal basis (called Fourier coefficients).  

\item Illustrate by examples that when a matrix is symmetric, eigenvectors corresponding to different eigenvalues are orthogonal, which means you can find an orthonormal basis to diagonalize the matrix. 

\end{enumerate}


Here are the preparation problems for this unit.

\begin{center}
\begin{tabular}{ll|l}
\multicolumn{2}{c}{Preparation Problems (\href{http://ilearn.byui.edu/bbcswebdav/institution/Physical\_Sci\_Eng/Mathematics/Personal\%20Folders/WoodruffB/341/7-Inner-Products-Preparation-Solutions.pdf}{click for handwritten solutions})}
%&
%Webcasts 
%(
%\href{http://ilearn.byui.edu/bbcswebdav/institution/Physical\_Sci\_Eng/Mathematics/Personal\%20Folders/WoodruffB/341/4-Linear-Transformations-videos.pdf}{pdf copy}
%)
\\
\hline\hline
Day 1& 7.2, 7.3, 7.4, 7.5
%&
%\href{http://ilearn.byui.edu/bbcswebdav/institution/Physical\_Sci\_Eng/Mathematics/Personal\%20Folders/WoodruffB/341/4-Linear-Transformations-video-01.wmv}{1},
%\href{http://ilearn.byui.edu/bbcswebdav/institution/Physical\_Sci\_Eng/Mathematics/Personal\%20Folders/WoodruffB/341/4-Linear-Transformations-video-02.wmv}{2},
%\href{http://ilearn.byui.edu/bbcswebdav/institution/Physical\_Sci\_Eng/Mathematics/Personal\%20Folders/WoodruffB/341/4-Linear-Transformations-video-03.wmv}{3}
\\ \hline
Day 2& 7.6, 7.9, 7.10, 7.12
%&
%\href{http://ilearn.byui.edu/bbcswebdav/institution/Physical\_Sci\_Eng/Mathematics/Personal\%20Folders/WoodruffB/341/4-Linear-Transformations-video-04.wmv}{4},
%\href{http://ilearn.byui.edu/bbcswebdav/institution/Physical\_Sci\_Eng/Mathematics/Personal\%20Folders/WoodruffB/341/4-Linear-Transformations-video-05.wmv}{5},
%\href{http://ilearn.byui.edu/bbcswebdav/institution/Physical\_Sci\_Eng/Mathematics/Personal\%20Folders/WoodruffB/341/4-Linear-Transformations-video-06.wmv}{6}
\\ \hline
Day 3& 7.14, 7.19, 7.20, 7.25
%&
%\href{http://ilearn.byui.edu/bbcswebdav/institution/Physical\_Sci\_Eng/Mathematics/Personal\%20Folders/WoodruffB/341/4-Linear-Transformations-video-07.wmv}{7},
%\href{http://ilearn.byui.edu/bbcswebdav/institution/Physical\_Sci\_Eng/Mathematics/Personal\%20Folders/WoodruffB/341/4-Linear-Transformations-video-08.wmv}{8},
%\href{http://ilearn.byui.edu/bbcswebdav/institution/Physical\_Sci\_Eng/Mathematics/Personal\%20Folders/WoodruffB/341/4-Linear-Transformations-video-09.wmv}{9}
\\ \hline
Day 4& 7.22, 7.26, 11.25, 11.27
\\ \hline
Day 5&
Lesson Plan,
Quiz, Prepare to teach each other
&
\\ \hline
\end{tabular}
\end{center}






The homework problems in this unit all come from Schaum's Oultines. 
\begin{center}
\begin{tabular}{|l|c|l|l|l|l|}
\hline
Concept&Sec.&Suggestions&Relevant Problems\\ \hline
Dot and Inner Product&7&1-5,7&1-7, 53-56\\ \hline
Orthogonality&7&9-14,&9-14, 60-67\\ \hline
Gram-Schmidt Process&7&18-23&18-23, 70-74\\ \hline
Orthogonal Matrices&7&24-28&24-28, 75-77\\ \hline
%Diagonalizing Symmetric Matrices&11&25, 27&25-30, 67-70\\ \hline
More with Orthogonal Matrices&3&&32-35, 97-100\\ \hline
\end{tabular}
\end{center}
Make sure you also review the linear regression problems from the second unit on Applications.  In this unit, our main focus is to learn how inner products are used to give us lengths, angles, projections, and then linear regression. Along the way we will pick up some important vocabulary which is used in future generalizations of these ideas to other subjects.  Orthonormal bases are extremely nice bases to use, since the corresponding matrix is orthogonal meaning $P^{-1}=P^T$ (inverting is simply transposing).



