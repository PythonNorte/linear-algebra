\chapter{Patterns}


The purpose of this chapter is to help you learn to create examples of patterns that others before you have noticed. Vector spaces are the most important pattern we will explore in this chapter.


\begin{enumerate}

\item Create examples to illustrate mathematical definitions. Know these definitions.
\item Create examples to illustrate theorems. You do not have to memorize all the theorems, rather when one is given make sure you can construct examples of it. 
\item When a matrix has an inverse, this is equivalent to many other facts about the matrix and corresponding systems of equations.  Illustrate this equivalence with examples, and be able to explain why the the ideas are equivalent.
\item Be able to explain why something is or is not a vector space. Give examples of vector spaces and subspaces that you have encountered previously in this class and other classes.
\item Explain how to find a basis for a vector space and the dimension of a vector space. Give the coordinates of vectors relative to a chosen basis.

\end{enumerate}


\section{Recognizing and Naming Patterns}

Have you ever noticed while reading a mathematical textbook that the book is filled with the following four things: (1) examples, (2) definitions, (3) theorems, and (4) proofs? This pattern follows throughout most of the sciences.  One of the main goals of mathematics is to describe the world around us, and these four things are precisely the tools used to do so.  Most of the time, recognizing a pattern starts with an example.  After working with lots of examples, you start to notice recurring patterns. Once you pinpoint exactly what that pattern is, you create a definition to explicitly describe the pattern. When you can describe exactly when and how a patterns appear, or if you can describe how different patterns relate to each other, you create a theorem. Proofs provide logical backbone to the theorems. Everything in this process starts with examples.  

Learning mathematics requires learning to work with examples. The main goal of this chapter is to learn how to create examples to understand the definitions and theorems that others have discovered.  If you encounter a new idea in any scientific area, start by looking at examples.  After you have studied several simple examples, you will discover patterns that help you understand the new idea.  Sometimes, your exploration of simple examples will result in new discoveries (theorems). Examining, understanding, and generalizing from examples is one of the main tools of mathematical and scientific research.  Learning to create and look at examples is crucial.

The most important pattern in linear algebra is the notion of a vector space. This is an abstract space that came from studying the properties of vectors in $\mathbb{R}^n$, and then realizing that these same properties hold in other places (like polynomials, functions, waves, and more). Before we formally define a vector space in general, we will practice creating examples to understand definition and theorems.  Most of what follows is a summary of the ideas introduced in the first 2 chapters. As you encounter each new definition or theorem, your homework will be to create an example which illustrates the new idea.  Once we have done this, we'll be ready to formally define vector spaces. Then the real power of linear algebra begins.


%
%Mathematical discoveries begin with examples. However, mathematical writing often begins with definitions. Those definitions are followed by theorems and proofs. If space permits, examples are often given to illustrate the definitions and theorems.  If appropriate, an application (example) is also given to show why a theorem is important. Those who write mathematical papers and textbooks around the nation have become accustomed to writing papers for peers in journals. Most of the time, these papers assume a lot of knowledge from the reader, and because of space limitations the papers are missing lots of examples. In order to read this kind of mathematics, you have to learn how to create examples to illustrate definitions and theorems whenever you see them.

%In this section, most of the definitions and theorems you have already seen and used in the previous chapters. The next three subsections present definitions, theorems, and then proofs. In the definitions and theorems section, I will provide examples of some of the ideas (or point you to examples from previous chapters). Your homework is to create examples to illustrate the definitions and theorems. In the proofs section, I will illustrate a few common techniques that are used to prove theorems.  The proofs in linear algebra are often shorter and easier to follow than proofs in other branches of mathematics. Those of you who plan to pursue any type of graduate study in engineering, physics, computer science, economics, statistics, and more will be expected to give proofs of the ideas you discover. In this book I will provide only some of the proofs and let you seek other references to find complete proofs to all the ideas. 



\subsection{Definitions}

Whenever you see a new definition, one of the most important things you can do is create an example which illustrates that definition. As you read each definition below, create an example which illustrates that definition. Most of these words are review, but a few are new.  Make sure you know these definitions.

This first definition is a formal statement of what we mean when we write $\mathbb{R}^n$.  It is the foundation for the formal vector space definition we'll see later.
\begin{definition} The symbol $\mathbb{R}$ is used to represent the real numbers.  
The \define{real vector space} $\mathbb{R}^n$ is a set together with a way of adding two elements (called \define{vector addition}) and a way of multiplying an element by a real number (called \define{scalar multiplication}).  
The set we use is the set of $n$-tuples $(v_1,v_2,\ldots, v_n)$ where each $v_i$ is an element of $\mathbb{R}$. 
We write this symbolically as 
$$
\{ (v_1,v_2,\ldots, v_n)\st v_i\in \mathbb{R} \text{ for } 1\leq i\leq n\}.$$
Each $n$-tuple $\vec v = (v_1,v_2,\ldots, v_n)$ is called a vector. 
The entries of the vector are called the \define[vector!components]{components} of the vector.  Vector addition is performed by adding the components of the vectors. Scalar multiplication is done by multiplying each component by the scalar. 
\end{definition}


\begin{definition} 
\marginpar{See example \ref{ex dot product}.}
\label{def dot product}
The \define{dot product} of two vectors $\vec u = \left<u_1,u_2,\ldots,u_n\right>$ and $\vec v =\left<v_1,v_2,\ldots,v_n\right>$ is the scalar $\vec u\cdot \vec v = u_1v_1+u_2v_2+\cdots+u_nv_n = \sum u_iv_i$. We say that two vectors are \define{orthogonal} if their dot product is zero.
\end{definition}

\begin{definition} \label{def matrix product} 
A (real) \define{matrix} is a rectangular grid of real numbers.
We use the notation $a_{ij}$ to represent the entry in the $i$th row and $j$th column.  
If a matrix $A$ has $m$ rows and $n$ columns, we say the matrix has \define[matrix!size]{size} $m$ by $n$ and write $A_{mn}$. 
The \define[matrix!sum]{sum} of two matrices of the same size is performed by adding corresponding entries.  
The \define[matrix!product]{product} of two matrices $A_{m n} = \left\{a_{ij}\right\}$ and $B_{np} =\left\{b_{jk}\right\}$ is a new matrix $C=AB$ where the $ik$th entry of $C$ is $c_{ik}=\sum_{j=1}^n a_{ij}b_{jk}$ (the dot product of the $i$th row of $A$ and the $k$th column of $B$).
\end{definition}

\begin{definition} \label{def rref}
\marginpar{See example \ref{ex rref last}. Every matrix we obtain in the row reduction process of $A$ is row equivalent to all the other matrices obtained from $A$ by doing row operations, and to the reduced row echelon form of $A$. Each of the matrices in example \ref{ex rref last} is row equivalent to all the other matrices in the example.}
We say that two matrices are \define[matrix!row equivalent]{row equivalent} if one can be obtained from the other by a finite sequence of these three row operations: (1) interchange two rows, (2) multiply one row by a nonzero constant, and (3) add to a row a multiple of a different row. 

A matrix is said to be in \define{row echelon form} (ref) if 
(1) each nonzero row begins with a 1 (called a \define{leading 1}), 
(2) the leading 1 in a row occurs further right than a leading 1 in the row above, and
(3) any rows of all zeros appear at the bottom.
The location in the matrix of each leading 1 is called a \define{pivot}, and the column containing the pivot is called a \define{pivot column}.
A matrix is said to be in \define{reduced row echelon form} (rref) if 
the matrix is in row echelon form, and each pivot column contains all zeros except for the pivot (the leading one).

\end{definition}

\begin{definition}[Homogeneous]
A linear system $A\vec x=\vec b$ is said to be either \define{homogeneous} if $\vec b=\vec 0$ or \define{non homogeneous} if $\vec b\neq \vec 0$.  
\end{definition}
\begin{definition}[Consistent]
A linear system is said to be \define{consistent} if it has at least one solution and  \define{inconsistent} if it has no solutions.
\end{definition}



\begin{definition}[Linear combination]\marginpar{See example \ref{ex span}.}%
A \define{linear combination of vectors} $\{\vec v_1,\vec v_2,\ldots,\vec v_n\}$ is a sum $$c_1 \vec v_1+c_2\vec v_2+\cdots+c_n\vec v_n = \sum_{i=1}^n c_i\vec v_i$$ where each scalar $c_1,c_2,\ldots, c_n$ is a real number. 
The \define{span} of a set of vectors $\{\vec v_1,\vec v_2,\ldots,\vec v_n\}$ is the set of all possible linear combinations of those vectors.
\end{definition}

\begin{definition}[Linear independence] \marginpar{See example \ref{first linearly independent example}. }%
We say a set $\{\vec v_1,\vec v_2,\ldots,\vec v_n\}$ of vectors is \define{linearly independent} if the only solution to the homogeneous equation $c_1 \vec v_1+c_2\vec v_2+\cdots+c_n\vec v_n=\vec 0$ is the trivial solution $c_1=c_2=\cdots=c_n=0$. If there is a nonzero solution to this homogeneous equation, then we say the set of vectors is \define{linearly dependent}, or we just say the vectors are linearly dependent (leave off ``set of'').
\end{definition}

\begin{definition}[Basis]
A \define{basis} for $\mathbb{R}^n$ is a set of vectors that are (1) linearly independent and (2) span all of $\mathbb{R}^n$.
The \define{standard basis} for $\mathbb{R}^n$ is the set 
$$\{\vec e_1 = (1,0,0,\ldots,0,0) , 
\vec e_2 = (0,1,0,\ldots,0,0) , 
\ldots, 
\vec e_n = (0,0,0,\ldots,0,1)\}.$$
\end{definition}

\begin{definition}[Column space] \marginpar{See examples \ref{colspace1ex} and \ref{colspace2ex}.}
The \define{column space} of a matrix is the span of the column vectors. 
A \define[column space!basis]{basis for the column space} of a matrix is a set of linearly independent vectors whose span is the column space. 
The \define[column space!dimension]{dimension} of the column space is the number of vectors in a basis for the column space. 
The \define{rank} of a matrix is the dimension of the column space.
\end{definition}


\begin{definition}[Row space] 
\marginpar{See examples \ref{colspace1ex} and \ref{colspace2ex}.}
The \textbf{row space} of a matrix is the span of the row vectors. 
A \textbf{basis for the row space} is a set of linearly independent vectors whose span is the row space. 
The \textbf{dimension} of the row space is the number of vectors in a basis for the row space.
\end{definition}

\begin{definition}
\marginpar{See examples \ref{colspace1ex} and \ref{colspace2ex}.}
If $\beta=\{\vec b_1,\vec b_2,\ldots,\vec b_n\}$ is a basis for a vector space, and a vector $\vec v$ equals the linear combination 
$$ \vec v = c_1\vec b_1+c_2\vec v_2 + \cdots c_n\vec v_n, $$ then we call the scalars $c_1,c_2,\ldots, c_n$ the  coordinates of $\vec v$ relative to the basis $\beta$. We may write $\vec v = (c_1,c_2,\ldots, c_n)_\beta$ to represent the vector $\vec v$, or if the basis $\beta$ is understood, we may just write $\vec v = (c_1,c_2,\ldots, c_n)$.
\end{definition}

\begin{definition} 
A \textbf{diagonal matrix} is a square matrix where the entries off the main diagonal are zero ($a_{ij}=0$ if $i\neq j$). The \textbf{identity matrix} is a diagonal matrix where each entry on the main diagonal is 1 ($a_{ii}=1$). We often write $I$ to represent the identity matrix, and $I_n$ to represent the $n\times n$ identity matrix when specifying the size is important. An \textbf{upper triangular matrix} is a square matrix where every entry below the main diagonal is zero. A \textbf{lower triangular matrix} is a square matrix where every entry above the main diagonal is zero.
$$\begin{array}{cccc}
\begin{bmatrix} 3&0&0\\0&-2&0\\0&0&8\end{bmatrix} 
&\begin{bmatrix} 1&0&0\\0&1&0\\0&0&1\end{bmatrix} 
&\begin{bmatrix} 1&3&2\\0&4&0\\0&0&-1\end{bmatrix}
&\begin{bmatrix} 1&0&0\\3&4&0\\2&0&-1\end{bmatrix} 
\\
\text{diagonal}
&\text{identity}
&\text{upper-triangular}
&\text{lower-triangular}
\end{array}$$
\end{definition}



\begin{definition} \marginpar{See example \ref{ex det}.}
The \textbf{determinant} of a $1\times 1$ matrix is the entry $a_{11}$. A \textbf{minor} $M_{ij}$ of an $n\times n$ matrix is the determinant of the $(n-1)\times (n-1)$ submatrix obtained by removing the $i$th row and $j$th column. A \textbf{cofactor} of a matrix is the product $C_{ij}=(-1)^{i+j}M_{ij}$. The determinant of an $n\times n$ matrix ($n\geq 2$) is found by multiplying each entry $a_{1i}$ on the first row of the matrix by its corresponding cofactor, and then summing the result. Symbolically we write  $$\det A = |A| = a_{11}C_{11}+a_{12}C_{12}+\cdots+a_{1n}C_{1n} = \sum_{j=1}^n a_{1j}C_{1j}.$$ 
\end{definition}


\begin{definition} \marginpar{See example \ref{ex inverse}}
The \textbf{inverse} of a square matrix $A$ is a matrix $B$ such that $AB=I$ and $BA=I$, where $I$ is the identity matrix. If a square matrix does not have an inverse, we say the matrix is \textbf{singular}.
\end{definition}

\begin{definition}  
\marginpar{See example \ref{ex transpose}.}
The \textbf{transpose} of a matrix $A_{m\times n}=\{a_{ij}\}$ is a new matrix $A^T_{n\times m}$ where the $i$th column of $A$ is the $i$th row of $A^T$.  A \textbf{symmetric matrix} is a matrix such that $A=A^T$.
\end{definition}


\begin{definition} \marginpar{See examples \ref{ex eigen1} and \ref{ex eigen2} .
Note that $\vec x=\vec 0$ is never an eigenvector.}
An \textbf{eigenvector} of a square matrix $A$ is a nonzero vector $\vec x$ with the property that $A\vec x = \lambda \vec x$ for some scalar $\lambda$.  We say that $\vec x$ is an \textbf{eigenvector} corresponding to the \textbf{eigenvalue} $\lambda$. 
\end{definition}





















%%%%%%%%%%%%%%%%Theorems
%%%%%%%%%%%%%%%%Theorems
%%%%%%%%%%%%%%%%Theorems
%%%%%%%%%%%%%%%%Theorems
%%%%%%%%%%%%%%%%Theorems
%%%%%%%%%%%%%%%%Theorems
%%%%%%%%%%%%%%%%Theorems
%%%%%%%%%%%%%%%%Theorems
%%%%%%%%%%%%%%%%Theorems
%%%%%%%%%%%%%%%%Theorems
%%%%%%%%%%%%%%%%Theorems
%%%%%%%%%%%%%%%%Theorems


\subsection{Theorems}

The definitions above summarize most of what we have done in the first two chapters. The list of properties below summarize many of the patterns we have already observed, together with many patterns that we will be exploring as the semester progresses.  Each is called a theorem because it is a statement that requires justification (a proof). In the homework, your job is to create examples to illustrate these theorems.  If a theorem is hard to remember, then create more examples to illustrate the theorem until you recognize the patterns that emerge.  Mathematics is all about finding patterns and then precisely stating the pattern in words. You do not have to memorize all of these theorems, but you should be able to create examples to illustrate them. In the next section, we will prove some of them.

Before stating the theorems, we first need to discuss the phrase ``if and only if.'' Theorem \ref{every column pivot iff independent} states 
\begin{quote}
The columns are linearly independent if and only if every column is a pivot column.
\end{quote}
We could write this statement symbolically by writing ``$p$ if and only if $q$.''
\marginpar{The statement ``$p$ if and only if $q$'' implies both 
\begin{enumerate}
	\item if $p$ then $q$, and 
	\item if $q$ then $p$.
\end{enumerate}
}
The words if and only if create two if then statements, namely (1) if $p$ then $q$ and (2) if $q$ then $p$. This gives us the two statments:
\begin{enumerate}
	\item If the columns are linearly independent then every column is a pivot column.
	\item If every column is a pivot column then the columns are linearly independent.
\end{enumerate}
Every time you see the words ``if and only if,'' remember that this means you have two if-then statements.  



\subsubsection{Vector Properties}
\begin{theorem}[Vector space properties of $\mathbb{R}^n$]\label{rn vector space properties}
Vector addition and scalar multiplication in $\mathbb{R}^n$ satisfy the following properties:
\begin{enumerate}
	\item[($A_1$)] Vector addition is associative: $(\vec u+\vec v)+\vec w = \vec u +(\vec v+\vec w)$.
	\item[($A_2$)] Vector addition is commutative: $\vec u+\vec v= \vec v+\vec u$.
	\item[($A_3$)] The zero vector $\vec 0$ satisfies $\vec 0+\vec v = \vec v+\vec 0=\vec v$.
	\item[($A_4$)] The additive inverse of $\vec v$ is the $-\vec v$ which satisfies $\vec v+(-\vec v)=0$.
	\item[($M_1$)] Scalar multiplication distributes across vector addition: $c(\vec u+\vec v)= c\vec u + c\vec v$.
	\item[($M_2$)] Scalar multiplication distributes across scalar addition: $(c+d)\vec v= c\vec v+ d\vec v$.
	\item[($M_3$)] Scalar multiplication is associative: $(cd)\vec v = c(d\vec v)$
	\item[($M_4$)] Scalar multiplication by 1 does nothing: $1\vec v=\vec v$
\end{enumerate}
\end{theorem}

\subsubsection{System Properties}
 
\begin{theorem}\label{thm unique rref}
Every matrix has a unique reduced row echelon form. This implies that for any matrix, the location of the pivot columns is unique. 
\end{theorem}

\begin{theorem}\label{thm row equivalent}
Two matrices are row equivalent if and only if they have the same reduced row echelon form.
\end{theorem}


\begin{theorem} \label{pivot columns iff} \label{thm solution iff augmented not pivot}
A linear system of equations $A\vec x=\vec b$ has a solution if and only if the number of pivot columns of $A$ equals the number of pivot columns of the augmented matrix $[A|b]$.
\end{theorem}

\begin{theorem}\label{thm no, one, or infinitely many solutions}
If a linear system of equations has more than one solution, it has infinitely many.  This means that for a linear system, there are three possible types of solutions: (1) no solution, (2) exactly one solution, or (3) infinitely many solutions. 
\end{theorem}

\begin{theorem}[Removing non pivot columns does not alter the row operations involved in obtaining rref.] Let $A$ be a matrix whose rref is $R$.  Let $B$ be the matrix obtained from $A$ by removing the $k$th column.  If the $k$th column of $A$ is not a pivot column, then the rref of $B$ is found by removing the $k$th column from $R$. 
\end{theorem}

\begin{theorem}\label{thm n-k free variables}
If a consistent linear system of equations $A\vec x=\vec b$ has $n$ unknowns and $k$ pivot columns, then it has $n-k$ free variables. 
\end{theorem}


\begin{theorem}[Superposition Principle] \label{thm superposition}\label{thm homogeneous}
For a homogeneous system $A\vec x = \vec 0$, any linear combination of solutions is again a solution.
\end{theorem}

\begin{theorem}\label{thm non homogeneous}
For a non homogeneous system $A\vec x=\vec b$, the difference $\vec x_1-\vec x_2$ between any two solutions is a solution to the homogeneous system $A\vec x=\vec 0$. 
\end{theorem}





\subsubsection{Matrix Properties}

\begin{theorem} [Vector space properties of $M_{mn}$]\label{matrix vector space properties}
Let $A,B$ and $C$ be $m$ by $n$ matrices, and $c,d$ be real numbers.  Matrix addition and scalar multiplication satisfy the following properties:
\begin{enumerate}
	\item[($A_1$)] Matrix addition is associative: $(A+B)+C = A +(B+C)$.
	\item[($A_2$)] Matrix addition is commutative: $A+B=B+A$.
	\item[($A_3$)] The zero matrix $0$ satisfies $0+A = A+0=A$.
	\item[($A_4$)] The additive inverse of $A$ is $-A$ which satisfies $A+(-A)=0$.
	\item[($M_1$)] Scalar multiplication distributes across matrix addition: $c(A+B)= cA + cB$.
	\item[($M_2$)] Scalar multiplication distributes across scalar addition: $(c+d)A= cA+ dA$.
	\item[($M_3$)] Scalar multiplication is associative: $(cd)A = c(dA)$.
	\item[($M_4$)] Scalar multiplication by 1 does nothing: $1A=A$.
\end{enumerate}
\end{theorem}

\begin{theorem}[Transpose of a product]
\label{thm transpose of product}
The transpose of a product is found by multiplying the transposes in the reverse order:
$$(AB)^T=B^T A^T.$$
\end{theorem}

\begin{theorem}
The product $A\vec x$ of a matrix $A$ and column vector $\vec x$ is a linear combination of the columns of $A$. 
\end{theorem}

\begin{theorem}\label{linearcombrow}
The product $\vec x A$ of a row vector $\vec x$ and matrix $A$ is a linear combination of the rows of $A$.
\end{theorem}

\begin{theorem}\label{matrix multiplication two ways}
Consider the matrices $A_{m\times n} = \begin{bmatrix}\vec a_1 & \vec a_2 &\cdots &\vec a_n\end{bmatrix}$ and 
$B_{n\times p} = \begin{bmatrix}\vec b_1 & \vec b_2 &\cdots &\vec b_p\end{bmatrix}$, where $\vec a_i$ and $\vec b_i$ are the columns of $A$ and $B$.  The product $AB$ equals $AB = \begin{bmatrix}A\vec b_1 & A\vec b_2 &\cdots &A\vec b_p\end{bmatrix}$. In other words, every column of $AB$ is a linear combination of the columns of $A$.
Similarly, every row of $AB$ is a linear combination of the rows of $B$.
\end{theorem}



\subsubsection{Linear Independence Properties}

\begin{theorem}\label{thm rank equal pivot columns}
The rank of a matrix is the number of pivot columns, which equals the number of leading 1's. 
\end{theorem}

\begin{theorem}\label{every column pivot iff independent}
The columns of a matrix are linearly independent if and only if every column is a pivot column.
\end{theorem}

\begin{theorem}\label{dependentiff}
A set (of 2 or more vectors) is linearly dependent if and only if at least one of the vectors can be written as a linear combination of the others. In particular, two vectors are linearly dependent if and only if one is a multiple of the other.
\end{theorem}


\begin{theorem}
The system $A\vec x=\vec b$ is consistent if and only if $\vec b$ is in the column space of $A$. 
\end{theorem}


\begin{theorem}\label{rankrowcolumn}
The dimension of the column space (the rank) equals the dimension of the row space.
%In particular, the pivot columns of a matrix form a basis for the column space and the nonzero row vectors in reduced row echelon form form a basis for the row space.
\end{theorem}


\begin{theorem}\label{thm colsp basis}
Consider a matrix $A$ whose columns are $\vec v_1,\vec v_2,\ldots,\vec v_n$. A basis for the column space is the set of pivot columns of $A$.  Let $R$ be the reduced row echelon form of $A$ with any rows of zeros removed. For each column $\vec v_k$ of $A$, the $k$th column of $R$ gives the coordinates of $\vec v_k$ relative to the basis of pivot columns.
  
In terms of matrix multiplication, if $C$ is the matrix obtained from $A$ by erasing the non pivot columns, and $\vec r_k$ is the $k$th column of $R$, then $C\vec r_k=v_k$. The columns of $R$ tell us how to linearly combine the pivot columns of $A$ to obtain all columns of $A$.
\end{theorem}

\begin{theorem}\label{thm rowsp basis}
Consider a matrix $A$ whose rows are $\vec v_1,\vec v_2,\ldots,\vec v_m$. Let $R$ be the reduced row echelon form of $A$ with any rows of zeros removed. A basis for the row space is the set of nonzero rows of $R$.  Let $C$ be the matrix obtained from $A$ by erasing the non pivot columns of $A$. For each row $\vec v_k$ of $A$, the $k$th row of $C$ gives the coordinates of $\vec v_k$ relative to the basis of nonzero rows of $R$.
  
In terms of matrix multiplication, if $\vec r_k$ is the $k$th row of $C$, then $\vec r_k R=v_k$. The rows of $C$ tell us how to linearly combine the rows of $R$ to obtain all rows of $A$.
\end{theorem}





\subsubsection{Inverse Properties}
\begin{theorem}\label{invunique}
The inverse of a matrix is unique.
\end{theorem}

\begin{theorem} \label{thm inverse of inverse}
The inverse of $A\inv$ is $A$. This implies that $(A^k)\inv=(A\inv)^k$ and $(cA)\inv = \frac{1}{c}A\inv$.
\end{theorem}

\begin{theorem}\label{thm inverse of transpose}
The inverse of the transpose is the transpose of the inverse: $(A^T)\inv = (A\inv)^T$.
\end{theorem}

\begin{theorem}[Socks-Shoes property]\label{invprod}
The inverse of a product is $(AB)\inv = B\inv A\inv$. Notice that the order of multiplication is reversed. You put on your socks and then your shoes. To undo this, you first take off your shoes, then your socks.
\end{theorem}

\begin{theorem}
If $A$ is invertible, then the solution to $A\vec x=\vec b$ is $\vec x=A\inv \vec b$.
\end{theorem}



\subsubsection{Determinant Properties}
\begin{theorem}
The determinant can be computed by using a cofactor expansion along any row or any column. Symbolically, we can write this as $\ds |A| = \sum_{i=1}^n a_{ij} C_{ij}$ for any column $j$,  or $\ds |A| = \sum_{j=1}^n a_{ij} C_{ij}$ for any $i$.
\end{theorem}


\begin{theorem}\label{thm det triangular}
The determinant of a triangular matrix is the product of its entries along the main diagonal.
\end{theorem}

\begin{theorem}\label{thm det product}
The determinant of $AB$ is $|AB|=|A||B|$. (Postpone the proof of this one.) This implies that $|cA|=c^n|A|$ and $|A\inv|=|A|\inv$. 
\end{theorem}

\begin{theorem}\label{thm det transpose}
The determinant of $A$ and $A^T$ are the same: $|A^T| = |A|$.
\end{theorem}

\begin{theorem}\label{thm det zero}
A matrix is invertible if and only if its determinant is not zero.
\end{theorem}

\begin{theorem}\label{thm det multiple columns}
If one column is a multiple of another column, then the determinant is zero. 
If one row is a multiple of another row, then the determinant is zero. 
\end{theorem}

\begin{theorem}\label{thm adjoint}
Let $A$ be a square matrix. Define the adjoint of $A$, written $\text{adj}(A)$ to be the matrix whose $ij$th entry is the $ji$th cofactor of $A$ (the transpose of matrix of cofactors).  
The product of the adjoint and the original matrix is $\text{adj}(A) A = |A|I$, a multiple of the identity matrix. 
If the determinant of $A$ is nonzero, this implies that the inverse of $A$ is $A^{-1}=\frac{1}{|A|}\text{adj}(A)$.
\end{theorem}




\subsubsection{Eigenvalues and Eigenvectors}
\begin{theorem}[How to compute eigenvalues and eigenvectors] \label{thm compute eigen}
The vector $\vec x$ is an eigenvector of $A$ corresponding to $\lambda$ if and only if $(A-\lambda I)\vec x=\vec 0$ and $\det(A-\lambda I)=0$.
\end{theorem}

\begin{theorem}\label{thm characteristic degree n}
For an $n$ by $n$ real matrix, the characteristic polynomial is an $n$th degree polynomial. This means there are $n$ eigenvalues (counting multiplicity and complex eigenvalues). 
\end{theorem}

\begin{theorem}\label{thm eigen triangular}
For a triangular matrix (all zeros either above or below the diagonal), the eigenvalues appear on the main diagonal.
\end{theorem}

\begin{theorem}\label{thm eigen transpose}
The eigenvalues of $A^T$ are the same as the eigenvalues of $A$. 
\end{theorem}

\begin{theorem}\label{thm eigenvalues and sums}
If the sum of every row of $A$ is the same, then that sum is an eigenvalue corresponding to the eigenvector $(1,1,\ldots,1)$. 
If the sum of every column of $A$ is the same, then that sum is an eigenvalue of $A$ as well (because it is an eigenvalue of $A^T$). 
A Markov process always has 1 as an eigenvalue because the columns sum to 1. 
\end{theorem}

\begin{theorem}\label{thm spectral}
If a matrix is symmetric ($A^T=A$), then the eigenvalues of $A$ are real. In addition, eigenvectors corresponding to different eigenvalues of a symmetric matrix are orthogonal (if $\vec x_1$ and $\vec x_2$ are eigenvectors corresponding to different eigenvalues, then $\vec x_1\cdot \vec x_2=0$).  
\end{theorem}

\begin{theorem}\label{thm independent eigenvectors}
Let $A$ be an $n$ by $n$ matrix.  Suppose there are $n$ distinct eigenvalues of $A$, named $\lambda_1, \lambda_2, \ldots, \lambda_n$. Let $\vec x_i$ be an eigenvector corresponding to $\lambda_i$ for each $i$.  The set $\{\vec x_1,\vec x_2, \ldots,\vec x_n\}$ is linearly independent, and hence is a basis for $\mathbb{R}^n$.
\end{theorem}











\subsection{Proving theorems}

In this section I'll illustrate various methods of justifying a theorem (giving a proof). You do not have to be able to prove every one of the theorems from the previous section. If you want a challenge, try to prove all of the theorems below.

\subsubsection{Prove by using a definition}
To prove theorem \ref{invprod} ($(AB)\inv=B\inv A \inv$), we just use the definition of an inverse. It often helps to rewrite a definition, perhaps changing the names of variables if needed. The inverse of a matrix $C$ is a matrix $D$ such that $CD=DC=I$ (multiplying on either side gives the identity). To show that the inverse of $AB$ is $B\inv A\inv$, we need to show that multiplying $AB$ on both the left and right by $B\inv A\inv$ results in the identity. We multiply $AB$ on the left by $B\inv A\inv$ and compute 
$$(B\inv A\inv)(AB) = B\inv (A\inv A)B = B\inv IB = BB\inv=I.$$
Similarly we multiply $AB$ on the right by $B\inv A\inv$ and compute 
$$(AB)(B\inv A\inv) = A(BB\inv) A\inv = AIA\inv = AA\inv=I.$$
 Because multiplying $AB$ on both the left and right by $B\inv A\inv$ gives us the identity matrix, we know $B\inv A\inv$ is the inverse of $AB$. All we did was use the definition of an inverse.

\subsubsection{Proving something is unique}
To show something is unique, one common approach is to assume there are two and then show they must be the same.  
We'll do this to show that inverses are unique (Theorem \ref{invunique}).  
Suppose that $A$ is a square matrix with an inverse $B$ and another inverse $C$.  
Using the definition, we know that $AB=BA=I$, and $AC=CA=I$.  
We now need to show that $B=C$ (show the two things are the same). 
If we multiply both sides of the equation $AB=I$ on the left by $C$, then we obtain $C(AB)=CI$. 
Since matrix multiplication is associative, we can rearrange the parentheses on the left to obtain $(CA)B=C$.  
Since $C$ is an inverse for $A$, we know $CA=I$, which means $IB=C$, or $B=C$, which shows that the inverse is unique.   



\subsubsection{Prove by using another theorem}
Sometimes using other theorems can quickly prove a new theorem.  
Theorem \ref{linearcombrow} says that the product $\vec x A$ is a linear combination of the rows of $A$.  
The two theorems right before this one state that $A\vec x$ is a linear combination of the columns of $A$, and that $(AB)^T=B^TA^T$.  
Rather than prove Theorem \ref{linearcombrow} directly, we are going to use these other two theorems.  
The transpose of $\vec x A$ is $A^T\vec x^T$, which is a linear combination of the columns of $A^T$.  
However, the columns of $A^T$ are the rows of $A$, so $A^T\vec x^T$ is a linear combination of the rows of $A$.  
This means that $\vec x A$ (undoing the transpose) is a linear combination of the rows of $A$. 
Theorems which give information about columns often immediately give information about rows as well.


\subsubsection{If and only if}
The words ``if and only if'' require that an ``if-then'' statement work in both directions.  The following key theorem (\ref{dependentiff}) has an if and only if statement in it. A set of 2 or more vectors is linearly dependent if and only if one of the vectors is a linear combination of others.  To prove this, we need to show 2 things.
\begin{enumerate}
	\item If the vectors are linearly dependent, then one is a linear combination of the others.
	\item If one of the vectors is a linear combination of the others, then the vectors are linearly independent.
\end{enumerate}
We'll start by proving the first item.  If the vectors are dependent, then there is a nonzero solution to the equation $$c_1 \vec v_1+c_2\vec v_2+\cdots+c_n\vec v_n=\vec 0.$$ Pick one of the nonzero constants, say $c_k$. Subtract $c_k\vec v_k$ from both sides and divide by $-c_k$ (which isn't zero) to obtain 
$$\frac{c_1}{-c_1}\vec v_1+\cdots+\frac{c_{k-1}}{-c_k}\vec v_{k+1}+\frac{c_{k+1}}{-c_k}\vec v_{k+1}+\cdots+\frac{c_n}{-c_k}\vec v_n=\vec v_k,$$
which means that $v_k$ is a linear combination of the other vectors.

Now let's prove the reverse condition. Suppose one of the vectors is a linear combination of the others (say the $k$th). 
Write 
$$v_k=c_1\vec v_1+\cdots+c_{k-1}\vec v_{k+1}+c_{k+1}\vec v_{k+1}+\cdots+c_n\vec v_n.$$  
Subtracting $v_k$ from both sides gives a nonzero solution ($c_k=-1$) to the equation $\vec 0=c_1\vec v_1+\cdots+c_{k-1}\vec v_{k+1}-v_k+c_{k+1}\vec v_{k+1}+\cdots+c_n\vec v_n$, which means the vectors are linearly dependent.




\subsubsection{Proof by contradiction}
Sometimes in order to prove a theorem, we make a contradictory assumption and then show that this assumption lead to an error (which means the assumption cannot be correct).  
We will do this to show that if $A$ is invertible, then $|A|\neq 0$. 
We will also use the fact that the determinant of a product is the product of the determinants, $|AB|=|A||B|$, a fact we will verify in a later chapter. 
Let's start with a matrix $A$ which has an inverse and then assume that the determinant is $|A|=0$.  
The inverse of $A$ is $A\inv$ and the product $AA\inv=I$ is the identity. 
Taking determinants of both sides gives $|AA\inv|=1$, or $|A||A\inv|=1$, since the identity matrix has determinant 1.  
If $|A|=0$, then $0|A\inv|=0=1$, which is absurd. 
Our assumption that $|A|=0$ must be incorrect. 
We must have $|A|\neq 0$.    

\subsubsection{Prove $a=b$ by showing $a\leq b$ and $b\leq a$}
If $a\leq b$ and $b\leq a$, then $a=b$. 
Sometimes it is easier to show an inequality between two numbers than it is to show the numbers are equal.  
We will use this to show that the rank of a matrix (the dimension of the column space - number of pivot columns) equals the dimension of the row space.
Let the rank of $A$ equal $r$, and let the dimension of the row space equal $s$. 
Write $A=CR$ where $C$ is the pivot columns of $A$ and $R$ is the nonzero rows of the RREF of $A$. 
The rank $r$ equals the number of columns in $C$. 
Since every row of $A$ can be written as a linear combination of the rows of $R$, the rows of $R$ span the rows space. 
We know that a basis for the row space cannot contain more than $r$ vectors since there are $r$ rows in $R$. 
This means that the dimension $s$ of the row space must statisfy $s\leq r$.
 
We now repeat the previous argument on the transpose of $A$. Write $A^T = C^\prime R^\prime$ where $C^\prime$ is the pivot columns of $A^T$ and $R^\prime$ is the nonzero rows of the RREF of $A^T$. The columns of $C^\prime$ are linearly independent rows of $A$, so $C^\prime$ has $s$ columns, and $R^\prime$ has $s$ rows. Since every row of $A^T$ can be written as a linear combination of the rows of $R^\prime$, a basis for the row space of $A^T$ (i.e. the column space of $A$) cannot have more than $s$ vectors, or $r\leq s$. Combining both inequalities, since $r\leq s$ and $s\leq r$, we have $r=s$.

%\subsubsection{Proofs involving $n$ elements}
%We have been using the idea that a non pivot column is a linear combination of the pivot columns preceding it (Theorem \ref{pivotcol}). To prove this, we need to work with summation notation. I'll start by repeating what we are trying to prove. Consider a matrix $A$ whose columns are $\vec v_1,\vec v_2,\ldots,\vec v_n$. Let $R$ be the reduced row echelon form of $A$ with any rows of zeros removed. If $\vec v_k$ is not a pivot column, then $\vec v_k$ is a linear combination of the preceding vectors. In particular, if $C$ is the matrix obtained from $A$ by erasing the non pivot columns, and $\vec r_k$ is the $k$th column of $R$, then $C\vec r_k=v_k$. In other words, the columns of $R$ tell us how to linearly combine the pivot columns of $A$ to obtain columns of $A$.
%
%Consider the equation $c_1 \vec v_1 +c_2 \vec v_2 +\cdots \vec c_n \vec v_n=0$ (in matrix form, just list the vectors and augment by zero). Since $v_k$ is not a pivot column, $c_k$ is a free variable. Let $c_k=-1$ and all other free variables equal zero. This allows us to write $c_1 \vec v_1 +c_2 \vec v_2 +\cdots+(-1)\vec v_k+\cdots+ \vec c_n \vec v_n=0$ or $$\vec v_k=c_1 \vec v_1 +c_2 \vec v_2 +\cdots+c_{k-1}\vec v_{k-1}+c_{k+1}\vec v_{k+1}+\cdots+ \vec c_n \vec v_n,$$ which means that $\vec v_k$ is a linear combination of the other vectors. Since every free variable is zero, we see that $v_k$ is a linear combination of the pivot columns of $A$. Because each free variable other that $c_k$ is zero, for all $i>k$ we have $c_i=0$ (non free variables only depend on the free variables which follow them).  This means that $v_k$ is linear combination of the vectors which precede it. To simplify notation below, let $C = \begin{bmatrix}\vec u_1 & \cdots &\vec u_m\end{bmatrix}$ be the matrix whose columns are the pivot columns of $A$.  Rename the constants above to obtain $\vec v_k = d_1\vec u_1 + \cdots +\d_m\vec u_m$, which corresponds to the augmented system $\begin{bmatrix}\vec u_1& \cdots &\vec u_m & \vec v_k\end{bmatrix}$ whose reduced row echelon form has all pivot columns except has $(d_1,\ldots,d_n)$ in the augmented system. 
%
%I need more here.  I will add some soon.  The connection between non pivot columns, linear combinations, and reduced row echelon form are crucial.


















\section{A huge equivalence - Connecting all the ideas}

\begin{theorem}\label{huge equivalence}
 For an $n$ by $n$ matrix $A$, the following are all equivalent (meaning you can put an if and only if between any two of these statements):
\begin{enumerate}
	\item The system $A\vec x=\vec 0$ has only the trivial solution.
	\item The system $A\vec x=\vec b$ has a unique solution for every $\vec b$.
	\item The reduced row echelon form of $A$ is the identity matrix.
	\item Every column of $A$ is a pivot column. 
	\item The columns of $A$ are linearly independent. (The dimension of the column space is $n$, and the span of the columns of $A$ is ${\mathbb{R}}^n$.)
	\item The rank of $A$ is $n$.
	\item The rows of $A$ are linearly independent. (The dimension of the row space is $n$, and the span of the rows of $A$ is ${\mathbb{R}}^n$.)
	\item The vectors $\vec e_1 = (1,0,0,\ldots,0), \vec e_2 = (0,1,0,\ldots,0),\ldots, \vec e_n = (0,0,0,\ldots,1)$ are in the column space of $A$.
	\item $A$ has an inverse.
	\item $|A|\neq 0$.
	\item The eigenvalues of $A$ are nonzero.	
\end{enumerate}
\end{theorem}

We have been using this theorem periodically without knowing it. Because this theorem is true, there is an if and only if between any pair of items.  In addition, because there is an if and only if between every item, we can also negate every single statement and obtain an equivalence as well:
\begin{theorem}
	There is more than one solution to  $A\vec x=0$ if and only if the system $A\vec x=\vec b$ does not have a unique solution (it may not have one at all), if and only if the reduced row echelon form of $A$ is not the identity matrix,  if and only if at least one column is not a pivot column,  if and only if the columns are linearly dependent (the dimension of the column space is less than $n$),  if and only if the rank is less than $n$,  if and only if the rows are linearly dependent,  if and only if there is no inverse ($A$ is singular),  if and only if the determinant is zero, and  if and only if zero is an eigenvalue.
\end{theorem}

The results of this theorem involve 2 results (an if and only if) between every pair of statements (making 110 different theorems altogether). We can prove the theorem much more quickly.  If we want to show 4 things are equivalent, then we can show that (1) implies (2), (2) implies (3), (3) implies (4), and (4) implies (1). Then we will have shown that (1) also implies (3) and (4) because we can follow a circle of implications to obtain an implication between any two statements.  To prove the theorem above we need to show a chain of 11 implications that eventually circles back on itself.

\begin{proof}

[(1) implies (2)] If the system $A\vec x=\vec 0$ has only the trivial solution, then the reduced row echelon form of $[A|\vec 0]$ is $[I|\vec 0]$. The row operations used to reduce this matrix will be the same as the row operations used to reduce $[A|\vec b]$. Hence the RREF of $[A|\vec b]$ will be $[I|\vec x]$, where $A\vec x=\vec b$, so there will be a unique solution.  

[(2)implies (3)]  With a unique solution to $A\vec x =\vec b$ for any $\vec b$, the solution to $A\vec x = \vec 0$ is $\vec x = \vec 0$.  This means that the RREF of $[A|\vec 0]$ is $[I|\vec 0]$. Since removing the last column of zeros won't change any row operations, the RREF of $A$ is $I$. 

[(3) implies (4)] Since the reduced row echelon form of $A$ is $I$, then each column contains a leading 1, and hence is a pivot column

[(4) implies (5)] Since each column is a pivot column, the only solution to $c_1 \vec v_1+c_2\vec v_2+\cdots+c_n\vec v_n=\vec 0$ (where $\vec v_i$ is the $i$th column of $A$) is the trivial solution $c_1=c_2=\cdots=c_n=0$. This is because in order for each column of $A$ to be a pivot column, each row of $A$ must have a leading 1 (since there are $n$ rows as well as $n$ columns).  Notice that we have just shown (4) implies (1) as well, which means that (1) through (4) are all equilvalent.

[(5) implies (6)] The definition of rank is the dimension of the column space. Since there are $n$ pivot columns, the rank is $n$.

[(6) implies (7)] This is the result of Theorem \ref{rankrowcolumn}, which also shows that (7) implies (5).

[(7) implies (8)] Since (7) is equivalent to (5), we know that the span of the column space is all of ${\mathbb{R}}^n$. This means in particular that the vectors listed are in the span of the columns of $A$.

[(8) implies (9)]  To find the inverse of $A$, we need to solve $A B = I$ or $[A\vec b_1\ A\vec b_2\ \ldots A\vec b_n]=[\vec e_1\ \vec e_2\ \ldots \vec e_n]$. This means we need to solve the equation $A\vec b_i=\vec e_i$ for $\vec b_i$ for each $i$, and this can be done because each $\vec e_i$ is in the column space of $A$. The inverse of $A$ is thus $[\vec b_1 \ \vec b_2 \ \ldots \vec b_n]$.

[(9) implies (10)] We already showed that if a matrix has an inverse, then the determinant is not zero (otherwise $|A||A\inv|=|I|$ gives $0=1$).

[(10) implies (11)] If the determinant is not zero, then $|A-0 I|\neq 0$, which means $0$ is not an eigenvalue.

[(11) implies (1)] If zero is not an eigenvalue, then the equation $A\vec x = 0\vec x$ cannot have a nonzero solution. This means that the only solution is the trivial solution.  
\end{proof}

At this point, if we want to say that (10) implies (9) (determinant nonzero implies there is an inverse), then the proof above shows that (10) implies (11) which implies (1) which implies (2) and so on until it implies (9) as well.  By creating a circle of implications, all of the ideas above are equivalent.


In the homework, I ask you to illustrate this theorem with a few matrices.  Your job is to show that the matrix satisfies every single one of the statements above, or to show that it satisfies the opposite of every single statement above. 




\section{Vector Spaces}

The notion of a vector space is the foundational tool upon which we will build for the rest of the semester. Vectors in the plane, vectors in space, and planes in space through the origin are all examples of vector spaces. We have looked at the column space, row space, null space, and eigenspace of a matrix as other examples of vector spaces.  In all of the examples we have looked at, we were able to create the vector space by considering a basis and using the span of that basis to obtain the vector space.  This is the fundamental idea we need to abstract vector spaces to a more general setting.  \marginpar{Vector spaces are spans of sets of vectors.}
Continuous and differentiable functions form vector spaces. Matrices of the same size form a vector space. Radio waves, light waves, electromagnetic waves, and more all form vector spaces. Mathematicians discovered some common properties about all of these kinds of objects, and created a definition of `Vector Space'' to allow them to focus on the key commonalities. The results which sprang from the following definition created linear algebra as we know it today. 

\begin{definition}\marginpar{We use the word ``real'' to emphasize that the scalars are real numbers. For those of you who know about fields, you can replace ``real'' with any field, such as the complex numbers or finite fields used in public key encryption. Scalar multiplication does not have to be done over the real numbers. For simplicity we will stick to the real numbers for now.
}
A (real) vector space is a set $V$ together with two operations:
\begin{enumerate}
	\item[(C1)] $+$ vector addition, which assigns to any pair $\vec u, \vec v\in V$ another vector $\vec u+\vec v\in V$,
	\item[(C2)] $\cdot$ scalar multiplication, which assigns to any $\vec v \in V$ and $c\in {\mathbb{R}}$ another vector $c\vec v$.
\end{enumerate}
We say that the set is closed under the operations of vector addition and scalar multiplication because adding vectors and scaling vectors produces vectors in the set. 
These two operations satisfy the following axioms ($c,d\in {\mathbb{R}}$ and $u,v,w\in V$):
\marginpar{Compare this to theorems \ref{rn vector space properties} and \ref{matrix vector space properties}.}
\begin{enumerate}
	\item[($A_1$)] Vector addition is associative: $(\vec u+\vec v)+\vec w = \vec u +(\vec v+\vec w)$.
	\item[($A_2$)] Vector addition is commutative: $\vec u+\vec v= \vec v+\vec u$.
	\item[($A_3$)] There is a zero vector $\vec 0$ in $V$ which satisfies $\vec 0+\vec v = \vec v+\vec 0=\vec v$.
	\item[($A_4$)] Every $\vec v\in V$ has an additive inverse, called $-\vec v$, which satisfies $\vec v+(-\vec v)=\vec 0$.
	\item[($M_1$)] Scalar multiplication distributes across vector addition: $c(\vec u+\vec v)= c\vec u + c\vec v$.
	\item[($M_2$)] Scalar multiplication distributes across scalar addition: $(c+d)\vec v= c\vec v+ d\vec v$.
	\item[($M_3$)] Scalar multiplication is associative: $(cd)\vec v = c(d\vec v)$
	\item[($M_4$)] Scalar multiplication by 1 does nothing: $1\vec v=\vec v$
\end{enumerate}
\end{definition}
The definition above is the formal definition of a vector space. The definition generalizes the properties we already use about vectors in 2 and 3 dimensions. This definition essentially means you can add and multiply as you would normally expect, no surprises will occur. 
 However, you can work with this definition if you think 
\begin{quote}
A vector space is a set of vectors where vector addition and scalar multiplication are closed, these operations obey the usual rules of addition and multiplication, and the span of any collection of vectors is also in the space.
\end{quote}


With vector addition and scalar multiplication properly defined, the span of any set of vectors will always be a vector space. Every linear combination of the vectors in that space must also be in the space. This means that lines and planes in 3D which pass through the origin are examples of vector spaces. As the semester progresses, we will discover that our intuition about what happens in 2D and 3D generalizes to all dimensions, and can help us understand any kind of vector space.  For now, let's look at some examples of vector spaces, and explain why they are vector spaces.

\begin{example} Examples of vector spaces:
\begin{enumerate}
	\item Both ${\mathbb{R}}^2$ and ${\mathbb{R}}^3$ are vector spaces. The collection of all $n$ dimensional vectors, ${\mathbb{R}}^n$, is a vector space.  Vector addition is defined component wise. Addition is associative, commutative, the zero vector is $(0,0,\ldots,0)$, and $(v_1,v_2,\ldots,v_n)+(-v_1,-v_2,\ldots,-v_n)=\vec 0$. Also, scalar multiplication distributes across vector and scalar addition, it is associative, and $1\vec v=\vec v$.  The definition of a vector space was defined to capture the properties of ${\mathbb{R}}^n$ which are useful in other settings.
	\item The set of $m$ by $n$ matrices, written $M_{mn}$, with matrix addition is a vector space.
	\item The set of all polynomials, written $P(x)$, is a vector space.
	\item The set of continuous functions on an interval $[a,b]$, written $C[a,b]$, where vector addition is function addition, is a vector space.
	\item The set of continuously differentiable functions on an interval $(a,b)$, written $C^1(a,b)$, where vector addition is function addition, is a vector space.
\end{enumerate}
\end{example}
\begin{example}
Examples that are not vector spaces:
\begin{enumerate}
	\item A line that does not pass through the origin. The zero vector is not on the line.
	\item Polynomials of degree $n$.  There is no zero vector.
	\item Invertible Matrices. There is no zero matrix.
	\item The nonnegative $x$ axis. There is no additive inverse for $(1,0)$, as $(-1,0)$ is not in the space.
	\item The line segment from $(-1,0)$ to $(1,0)$. The product $2(1,0)=(2,0)$ is not in the space.
\end{enumerate}	
\end{example}


\begin{example} [The set of all polynomials is a vector space]
To illustrate how to verify that a set is a vector space, let's formally verify that one of the examples above is indeed a vector space. Let's work with the set of all polynomials $P(x)$. Our vectors are polynomials, which are functions of the form
$$a(x) = a_0 +a_1 x+a_2 x^2 + \cdots + a_n x^n = \sum_{i=0}^n a_i x^i.$$
We must define vector addition and scalar multiplication.  
If $\ds a(x) =  \sum_{i=0}^n a_i x^i$ and $\ds b(x) =  \sum_{i=0}^m b_i x^i$ are two polynomials, define their sum to be a new polynomial $c(x)$ where $c_i = a_i+b_i$ for each $i$ (where $a_i=0$ if $i> n$ and $b_i=0$ if $i> m$). 
For example, we compute the sum $(1+4x+3x^2)+(x-x^3) = (1+0)+(4+1)x+(3+0)x^2+(0-1)x^3$. If $c$ is a real number, then scalar multiplication $c\ a(x)$ is a new polynomial $b(x)$ where $b_i=ca_i$ for each $i$.   

We now need to illustrate the 8 axioms hold:
\begin{enumerate}
	\item[($A_1$)] Polynomial addition is associative: $(a(x)+b(x))+c(x) = a(x) +(b(x)+c(x))$. To verify this, we look at each coefficient of the sum.  The coefficients are $(a_i+b_i)+c_i = a_i+(b_i+c_i)$, and equality holds because addition of real numbers is associative.
	\item[($A_2$)] Polynomial addition is commutative: $a(x)+b(x)=b(x)+a(x)$. Again this follows because addition of real numbers is commutative. In particular, $a_i+b_i=b_i+a_i$ for each $i$.
	\item[($A_3$)] The zero polynomial $z(x)=0$ is the zero vector, as $0+a(x) = a(x)+0=a(x)$ for any polynomial.
	\item[($A_4$)] The additive inverse of $a(x) = a_0+a_1x+\cdots+a_nx^n$ is $-a(x) = -a_0-a_1x-\cdots-a_nx^n$.
	\item[($M_1$)] Scalar multiplication distributes across vector addition: $c(a(x)+b(x))= ca(x) + cb(x)$. For each $i$, we compute $c(a_i+b_i) = ca_i+cb_i$. 
	\item[($M_2$)] Scalar multiplication distributes across scalar addition: $(c+d)a(x)= ca(x)+ da(x)$. For each $i$, we compute $(c+d)a_i = ca_i+da_i$. 
	\item[($M_3$)] Scalar multiplication is associative: $(cd)a(x) = c(da(x))$. For each $i$, we compute $(cd)a_i = c(da_i)$.
	\item[($M_4$)] Scalar multiplication by 1 does nothing: $1a(x)=a(x)$. For each $i$, we compute $1a_i = a_i$.
\end{enumerate}
We have now shown that the set of polynomials is a vector space.
\end{example}

\subsection{Subspaces}
The formal definition of a vector space is rather long. Trying to show that a set is a vector space can be a tedious computation. Once we have taken the time to show that a set is a vector space, we can use this result to find many subspaces inside the vector space that are themselves vector spaces.  We have done this already when we looked at the column space, row space, null space, and eigenspaces of a matrix. The notion of a subspace allows us to quickly find new vector spaces inside of some we already understand.

\begin{definition}
Suppose $V$ is a vector space. If $U$ is a set of vectors in $V$ such that $U$ is a vector space itself (using the addition and multiplication in $V$), then we say that $U$ is a subspace of $V$ and write $U\subset V$.   
\end{definition}

As an example, we know that the vectors in the plane form a vector space. Every line through the origin forms a vector subspace of the plane. Every line and plane in 3D which passes through the origin forms a vector subspace of ${\mathbb{R}}^3$. We now state a key theorem which allows us to quickly verify that a set of vectors in a vector space is a vector subspace.

\begin{theorem}\label{thm subspace iff closed}
Suppose $U$ is a set of vectors in a vector space $V$ and the following three conditions hold:
\begin{enumerate}
  \item $0\in U$
	\item $\vec u+\vec v\in U$ for every $\vec u,\vec v\in U$ (the set is closed under vector addition)
	\item $c\vec u\in U$ for every $\vec u\in U$ and $c\in {\mathbb{R}}$ (the set is closed under scalar multiplication)
\end{enumerate}
Then $U$ is a vector subspace of $V$. 
\end{theorem}
The first item requires that the set actually contain at least one element.  
The second and third items basically state that every linear combination of vectors in $U$ is also in $U$ (the set is closed under linear combinations). 
A vector subspace is a nonempty set of vectors in which every linear combination of vectors is also in the set. 
The key value of this theorem is that it allows us to quickly show that many spaces are vector spaces.  
Let's start with some examples, and then we'll end with a proof.



\begin{example} Here are some examples of vector subspaces.
\begin{enumerate}
	\item  The span of a set of vectors is always a subspace of the larger vector space. This is because every linear combination is by definition in the subspace.  Hence the row space and column space are subspaces (as they are defined as spans or columns and rows).
	\item  The set of upper triangular $m$ by $n$ matrices, is a subspace of $M_{mn}$. The zero matrix is in $M_{mn}$. The sum of two upper triangular matrices is upper triangular.  Multiplying each entry of an upper triangular matrix will still be upper triangular. 
	\item  The set of polynomials of degree less than or equal to $n$, written $P_n(x)$, is a subspace of $P(x)$. The zero polynomial (degree zero) is in the set.  If two polynomials start out with degree less than or equal to $n$, then their sum cannot increase in degree.  In addition, multiplying a polynomial by a constant cannot increase the degree.
	\item  More examples are given in Schaum's outlines on page 160 (examples 4.4, 4.5, 4.6). 
\end{enumerate}
\end{example}

\begin{example}
Here are some examples of a set that is not a vector subspace. To show something is not a vector subspace, find one example that shows the set is not closed under addition or scalar multiplication.
\begin{enumerate}
	\item The set of vectors on the positive $x$ axis ($V=\{(a,0)|a\geq0\}$) is not a vector subspace because you cannot multiply any of them by $-1$ (not closed under scalar multiplication). To be a vector subspace, once you are allowed to move in one direction, you have to be able to also move in the opposite direction.
	\item The set of vectors in 2D whose magnitude is less than or equal to one (so any vector ending inside the unit circle) is not a vector subspace.  One reason why is that you can't multiply $(1/2,0)$ by 4, as $(2,0)$ is outside the circle (not closed under scalar multiplication).  To be a vector subspace, once you are allowed to move in one direction, you should be able to move in that direction forever.
	\item The vectors in 2D which are on the $x$ or $y$ axis is not a vector subspace. The sum $(1,0)+(0,1) = (1,1)$ is not on either axis (not closed under vector addition).  To be a vector subspace, you must be allowed to add any two vectors and still obtain a vector in the space.
	\item The set of polynomials of degree 2 is not a vector subspace of the set of polynomials, because the zero polynomial is not in this set. Also, the sum of $x^2$ and $1-x^2$ is $(x^2)+(1-x^2)=1$, which is a polynomial of degree zero instead of degree 2 (not closed under addition).
\end{enumerate}
\end{example}





\begin{proof}
The proof of this theorem is rather quick. Let $V$ be the larger vector space, and $U$ the subset of vectors that satisfy the 3 properties.  We are assuming that vector addition and scalar multiplication give us vectors in $U$. Because every vector in $U$ is also a vector in $V$, we immediately know that vector addition is associative ($A_1$) and commutative ($A_2$), and that scalar multiplication distributes across vectors ($M_1$) and scalar addition ($M_2$), is associative ($M_3$), and multiplication by 1 does nothing ($M_4$). The first property assumes there is a zero vector ($A_3$). So all we need to show is that there are additive inverses ($A_4$). If $\vec u\in U$, then because $c\vec u\in U$ for every $c\in \mathbb{R}$ we can pick $c=-1$ to obtain $-1 \vec u\in U$. This means that $u+(-1\vec u) = (1-1)\vec u = 0\vec u = \vec 0$, which means every $u\in U$ has an additive inverse $-1\vec u\in U$ ($A_4$). We have now verified that $U$ satisfies all the axioms required to be a vector space, so this complete the proof. 
\end{proof}


The first example we looked at of a vector subspace involved looking at the span of a set of vectors. Because column and row spaces are defined as spans of vectors, they are vector subspaces. Because any vector subspace is closed under linear combinations, the span of the vector subspace is itself. This means that every vector subspace is the span of a set of vectors. This pattern will be used so often that we will call it a theorem.  

\begin{theorem}\label{thm subspace iff span}
If $W$ is a subset of vectors in a vector space $V$, then $W$ is a subspace of $V$ if and only if $W$ is the span of a set of vectors.
\end{theorem}


\begin{example}
Let's use this theorem to show that some sets are vector subspaces. \marginpar{If you can show a set is spanned by a set of vectors, then the set must be a vector space.}
\begin{enumerate}
	\item The set of 2 by 2 diagonal matrices is spanned by 
	$\begin{bmatrix}1&0\\0&0\end{bmatrix}$ and $\begin{bmatrix}0&0\\0&1\end{bmatrix}$, hence a vector subspace of $M_{22}$.
	\item We already know that $P(x)$ is a vector space. The span of $\{1,x,x^2,x^3,x^4\}$ is the set of polynomials of the form $a+bx+cx^2+dx^3+ex^4$, which is all polynomials of degree 4 or less, $P_4(x)$. Since $P_4(x)$ is the span of set of vectors, it is a vector subspace of $P(x)$. 
	\item  
\end{enumerate}
\end{example}



\subsection{Basis and Dimension}
Vector spaces are often rather large sets. In the previous chapter, we found basis vectors for the column and row space of a matrix by row reducing a matrix. These basis vectors allow us to talk about the entire space by considering the span of the basis vectors. The coordinates of other vectors can then be given relative to the chosen basis. Let's review these ideas with an example of finding a basis for the column space of a matrix.

\begin{example}
The reduced row echelon form of 
$\begin{bmatrix}[ccc] 1&-1&1\\3&0&6\\5&1&11\end{bmatrix}$ 
is 
$\begin{bmatrix}[ccc] 1&0&2\\0&1&1\\0&0&0\end{bmatrix}$. 
The column space is the vector space spanned by the column vectors 
$\vec u_1= (1,3,5))$, 
$\vec u_2= (-1,0,1)$, and 
$\vec u_3= (1,6,11)$.
The reduced row echelon form tells us that the third column is 2 times the first plus the second.   
Since the third column is already a linear combination of the first two columns, it does not contribute any new vectors to the span of just the first two vectors.  
Hence, the column space is the span of $\vec u_1$ and $\vec u_2$ (we can remove the dependent vector without shrinking the number of vectors in the span). 
Because the first two columns are linearly independent, we cannot remove either without reducing the number of vectors in their span.

A basis for the column space is a linearly independent set of vectors whose span is the the column space. 
Because $\vec u_1$ and $\vec u_2$ are linearly independent, and they span the column space, they form a basis for the column space. 
The coordinates of the third vector relative to this basis are $(2,1)$, the numbers in the third column of the rref.  
The dimension of the column space is the number of vectors in a basis for the column space, so here the dimension of the column space is 2.  The column space is a plane of vectors in space.
\end{example}

The example above illustrates the key ideas we need to work with any finite dimensional vector spaces.  Let's now give the formal definitions.

\begin{definition}
A \textbf{basis for a vector space} is a set of vectors that are 
(1) linearly independent and (2) span the vector space.
The \textbf{dimension of a vector space} is the number of vectors in a basis for the vector space. 
We say the dimension of the zero vector space (the vector space consisting of only the zero vector) is zero. 
If $\beta=\{\vec v_1,\vec v_2,\ldots,\vec v_n\}$ is a basis and $\vec v = c_1\vec v_1+c_2\vec v_2+\cdots+c_n\vec v_n$, then we call $(c_1,c_2,\ldots,c_n)_\beta$ the \textbf{coordinates of $\vec v$ relative to the basis $\beta$}. 
If the basis is understood, we just leave off the subscript $\beta$.
\end{definition}

Remember, once you have a new definition, you should always try to illustrate that definition with examples.  Let's now look at a few more examples. 
\begin{example} Let's start by looking at $\mathbb{R}^n$, a vector space we are familiar with.
The standard basis for ${\mathbb{R}}^2$ is $\vec e_1 = (1,0),\vec e_2=(0,1)$, so the dimension is 2. 
Similarly,  $\vec e_1 = (1,0,\ldots,0),\vec e_2=(0,1,\ldots,0), \ldots, \vec e_n=(0,0,\ldots,1)$ is a basis for ${\mathbb{R}}^n$, so the dimension is $n$.  
\end{example}

\begin{example}
Let's find a basis for a polynomial vector space. 
Consider the set of polynomials $\{1, x, x^2\}$. The span of these polynomials is $a_0(1)+a_1(x)+a_2(x^2)$, which is every polynomial of degree less than or equal to 2. 
To show that this set forms a basis for $P_2(x)$, we only need to show that these polynomials are linearly independent.  
To show this, we must show that $$a_0(1)+a_1(x)+a_2(x^2) = 0 \text{ implies } a_0=a_1=a_2=0$$ (the only linear combination which results in the zero polynomial is the trivial combination).  
Since the equation above holds for every real number $x$, we can simply plug in 3 different $x$ values to obtain 3 different equations. Letting $x=0,1,-1$ gives us the three equations $a_0=0$, $a_0+a_1+a_2=0$, and $a_0-a_1+a_2=0$. Solving this system yields $a_0=a_1=a_2=0$. Hence, the three polynomials $1, x, x^2$ are linearly independent.  
These three polynomials form a basis for $P_2(x)$, and we call this the standard basis for $P_2$.  Using this basis, the coordinate of $a_0+a_1x+a_2x^2$ relative to the standard basis are $(a_0,a_1,a_2)$.
Notice that dimension of $P_2(x)$ is 3 (one more than the degree of the polynomial). 
\marginpar{In general, the set $\{1, x, x^2, x^3, \ldots, x^n\}$ is called the standard basis for $P_n(x)$.} 
\end{example}


For vector spaces that are not ${\mathbb{R}}^n$, a simple way to work with vectors is to start by finding a basis. Once you have a basis, you can give the coordinates of every vector in the space relative to that basis.  These coordinates can then be thought of as vectors in ${\mathbb{R}}^n$. Questions about the vector space can then be reduced to questions about vectors in $\mathbb{R}^n$.  As seen in the last example, the vector space $P_2(x)$ consists of all polynomials of the form $a_0+a_1x+a_2x^2$. If we use the standard basis for polynomials, then the coordinates are simply $(a_0,a_1,a_2)$. We can now use these coordinates to ask other questions about the vectors in $P_2(x)$. Here are two examples.

\begin{example}
Let $V$ be the vector space spanned by $1+x,x+x^2, 4+x-3x^2$. We'll show it is a 2 dimensional vector subspace of $P_2(x)$, and provide a basis for this space in 2 different ways. The coordinates of these 3 polynomials relative to the standard basis are $(1,1,0)$, $(0,1,1)$, and $(4,1,-3)$. 
To study the span of the polynomials, we instead look at the column space of 
$$
A=
\begin{bmatrix}
 1 & 0 & 4\\
 1 & 1 & 1\\
 0 & 1 & -3
\end{bmatrix} 
\xrightarrow{rref}
\begin{bmatrix}
 1 & 0 & 4\\
 0 & 1 & -3\\
 0 & 0 & 0
\end{bmatrix} 
.$$
Row reduction of $A$ shows us that the first two columns of $A$ are pivot columns, 
hence basis vectors for the column space of $A$. 
This tells us that the corresponding set of polynomials $\{1+x, x+x^2\}$ is a basis for $V$. 
The coordinates of $4+x-3x^2$ relative to this basis are $(4,-3)$.

To find a different basis for $V$, we instead look at the row space of
$$A^T
=
\begin{bmatrix}
 1 & 1 & 0 \\
 0 & 1 & 1 \\
 4 & 1 & -3
\end{bmatrix}
\xrightarrow{rref}
\begin{bmatrix}
 1 & 0 & -1 \\
 0 & 1 & 1 \\
 0 & 0 & 0
\end{bmatrix}
.$$ 
Row reduction tell us that $(1,0,-1)$ and $(0,1,1)$ are basis vectors for the row space of $A^T$ (hence column space of $A$).  The corresponding polynomials $1-x^2$ and $x+x^2$ are basis vectors for $V$. The coordinates of $4+x-3x^2$ relative to this basis are $(4,1)$.
  
\end{example}

\begin{example}
Let's show the vectors $1 = (1,0,0),1+x=(1,1,0), 1+x+x^2=(1,1,1)$ form a basis for $P_2(x)$, and then write $2+3x-4x^2 = (2,3,-4)$ as a linear combination of these vectors. 
To do this, we place the coordinates of each vector (relative to the standard basis) into the columns of a matrix, and then reduce 
$$\begin{bmatrix}
 1 & 1 & 1 & 2 \\
 0 & 1 & 1 & 3 \\
 0 & 0 & 1 & -4
\end{bmatrix}
\xrightarrow{rref}
\begin{bmatrix}
 1 & 0 & 0 & -1 \\
 0 & 1 & 0 & 7 \\
 0 & 0 & 1 & -4
\end{bmatrix}.$$ 
Since the first 3 columns are pivot columns, they are linearly independent. 
This implies that the polynomials represented by these vectors are linearly independent. Since the three sets of coordinates span all of $\mathbb{R^3}$, the three polynomials must span all of $P_2(x)$.
The last column of rref tells us that the coordinates of $(2,3,-4)$ relative to $\beta=\{(1,0,0),(1,1,0),(1,1,1)\}$ are $(-1,7,-4)_\beta$. In terms of polynomials we write $2+3x-4x^2 = -1(1)+7(1+x)-4(1+x+x^2)$. To emphasize the vocabulary, we can now state two things:
\begin{enumerate}
	\item The coordinates of $2+3x-4x^2$ relative to $\{1,x,x^2\}$ are $(2,3,-4)$.
	\item The coordinates of $2+3x-4x^2$ relative to $\{1,1+x,1+x+x^2\}$ are $(-1,7,-4)$.
\end{enumerate}
\end{example}

We'll often start by using a standard basis to represent vectors, and then change the basis to a different one if the need arises.  In chapter 5 we will explore this idea in more depth.  For now we'll focus on finding a basis and coordinates relative to that basis.

\begin{example}
As a final example, let's introduce the standard basis for the vector space of $m$ by $n$ matrices. We'll focus on 2 by 2 matrices, though this example generalizes to any size matrix. The set of matrices 
$$\left\{
\begin{bmatrix}
 1 & 0  \\
 0 & 0 
\end{bmatrix},
\begin{bmatrix}
 0 & 1  \\
 0 & 0 
\end{bmatrix},
\begin{bmatrix}
 0 & 0  \\
 1 & 0 
\end{bmatrix},
\begin{bmatrix}
 0 & 0  \\
 0 & 1 
\end{bmatrix}
\right\}$$
forms a basis for the vector space $M_{22}$ of 2 by 2 matrices (which means this space has dimension 4).  
To show it is a basis, we must show that (1) the matrices are linearly independent and (2) the matrices span $M_{22}$.
To show linearly independent, we solve 
$$c_1
\begin{bmatrix}
 1 & 0  \\
 0 & 0 
\end{bmatrix}
+c_2
\begin{bmatrix}
 0 & 1  \\
 0 & 0 
\end{bmatrix}
+c_3
\begin{bmatrix}
 0 & 0  \\
 1 & 0 
\end{bmatrix}
+c_4
\begin{bmatrix}
 0 & 0  \\
 0 & 1 
\end{bmatrix}
=
\begin{bmatrix}
 c_1 & c_2  \\
 c_3 & c_4 
\end{bmatrix}
=\begin{bmatrix}
 0 & 0  \\
 0 & 0 
\end{bmatrix}
.$$
The only solution is the trivial solution, so the matrices are linearly independent. 
These 4 matrices clearly span the space, as the coordinates of any matrix 
$\begin{bmatrix}
 a & b  \\
 c & d 
\end{bmatrix}$ 
relative to this basis are $(a,b,c,d)$.
We can now work with 2 by 2 matrices as a vector space by considering vectors in ${\mathbb{R}}^4$ instead. 
\end{example}

Whenever we encounter vector spaces of finite dimension, we will often start by finding a basis and then work with coordinates relative to this basis instead. Then we can work with the coordinates as vectors in ${\mathbb{R}}^n$ to perform calculations. This means that once we understand what happens in ${\mathbb{R}}^n$, we can understand what happens in any vector space of dimension $n$. 


%%% Local Variables: 
%%% mode: latex
%%% TeX-master: "../linear-algebra"
%%% End: 
