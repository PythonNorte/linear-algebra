\section{Notes for teaching}

Creating examples to illustrate a definition is like describing a fence.  To nail down the definition, you need to illustrate it both with something that satisfies the definition, as well as something that does \emph{not} satisfy the definition (but close).  Think of it as putting two stakes down on either side of a fence.  The fence then won't move, and you know exactly the area it describes.

\subsection{Ben Woodruff}
\label{sec:ben-woodruff}


The last time I taught it, I spoke about patterns the whole first day. It was long and dry.  I showed them how to create an example to illustrate a theorem, and talked about what I expected them to do in the homework.  I won't do this first next time.  I'd do what I did the second day.

The second day I put up a few simple matrices, and then I had them compute a bunch of things and then try to express what they discovered as a pattern. For example, I had them compute $|A|$, $|B|$, and $|AB|$, or $|A|$ and $|A^T|$ (which leads directly to why the eigenvalues are the same).  I had them compute $(A^T)^{-1}$ and $(A^{-1})^T$.  I had them compute the eigenvalues of $A$ and $A^T$.  I had them compute $A^{-1}$, $B^{-1}$, $(AB)^{-1}$, $A^{-1}$ $B^{-1}$ and $B^{-1}$ $A^{-1}$ (obviously keep this small and use facts about 2 by 2 matrices to make it go quick). I went through all the theorems and select the key ones I wanted to spend time on in class.  This was a whole lot more effective than me standing up at the board and showing examples.  You could prove some of these theorems, but for many of the theorems you can actually see how the proof would work straight from the 2 by 2 example.  (for $|A|=|A^T|$ I think using a 3 by 3 example is crucial to see why they are the same, together with a hint that as they work the example, they should always cofactor along rows in one, and columns in the other.) Once the students had each worked through 4 of the 8 theorems, we stopped working at the boards and then I went around the room and took a minute or two at each one.  (I had each wall responsible for 2 of the theorems, and once they finished their two they were supposed to move on and work through examples of some others). We got through 8 theorems, with examples and some proofs in 30 minutes.  For homework the students are supposed to create their own examples.  Make sure you let them know that this will require that they read, and learn to use definitions.  It takes time and practice, and working together is a good idea.

The last half of that second day I spent on the proof of the huge equivalence.  Because we haven't been spending time proving every detail, it didn't go over very well.  Next time I plan to pick two matrices and have them work through this equivalence and then state the theorem themselves.  Then we will quickly talk about why the ideas are equivalent.

Then you get to start into Vector Spaces.  This will just take a few days period to get through the idea, as well as subspaces.  Have them watch the webcasts before coming to class.

I would try to get to examples involving bases and coordinates asap. You may even find that going through vector spaces and subspaces quickly, hitting bases and coordinates, and then going back might be better than just waiting till everyone is on board with Vector Spaces.

Since every vector space is the span of some vectors, I'm starting to wonder if introducing it that way first would be appropriate (talking about bases and spans first, showing some weird examples where we change how addition works, and then pinpointing the axioms).  For now, I'll leave it axiomatic (but this isn't intuitive).

I did spend time in class working with the logarithmic vector space (which is one of the webcasts), and I had the students work in groups a bunch to decide if sets were vector subspaces (I never sent them to the boards to determine if something was a vector space using all 10 axioms).  I did focus on polynomial and functions as the key examples of vector spaces.
